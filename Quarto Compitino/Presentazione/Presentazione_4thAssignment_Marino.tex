\documentclass[10pt,xcolor={table,dvipsnames}]{beamer} 		% carica automaticamente amsthm, amssymb, amsmath, graphicx

\usepackage[T1]{fontenc}
%\usepackage[utf8]{inputenc}
\usepackage[italian]{babel}

\usepackage{mathtools}					% amsmath sotto steroidi
\usepackage{mathdots}
\usepackage{physics}

\usepackage{bm}
%\usepackage{anyfontsize}

\usepackage{xparse}

\usepackage{mathrsfs}					% Per dei caratteri matematici migliori: \mathscr{} e \mathcal{}
%\usepackage{braket} 					% Per il comando \Set, e altre (poche) cose
\usepackage[italian]{varioref}			% Per usare il comando \vref{label}, che dà dei collegamenti più dettagliati
\usepackage{microtype}					% Migliora la tipografia, permettendo ad alcuni elementi di sporgere leggermente
%\usepackage{textcomp}					% Dovrebbe aggiungere più simboli

\usepackage{relsize}					% Per usare \mathbigger{} ecc

\usepackage{multirow}					% per usare il comando multirow
\usepackage{tabularx}					% per fare tabelle. Carica il pacchetto array, per gli array.
%\usepackage{arydshln}					% per le linee tratteggiate nelle tabelle

%\usepackage[many]{tcolorbox}


\DeclarePairedDelimiter{\absval}{\lvert}{\rvert}
\DeclarePairedDelimiter{\norma}{\lVert}{\rVert}

%\setcounter{tocdepth}{1}	% profondità dell'indice

	% TEOREMI CUSTOM:
\theoremstyle{plain}					% Definisce ambienti per Teoremi, esercizi, corollari... Con lo stile adeguato
	\newtheorem{proposizione}{Proposizione}[section]
	\newtheorem*{proposizione*}{Proposizione}
	
	\newtheorem{teorema}{Teorema}[section]
	\newtheorem*{teorema*}{Teorema}
		
	%\newtheorem{lemma_es}{Lemma}[esercizio]
	%\newtheorem{lemma}{Lemma}[section]
	\newtheorem*{lemma*}{Lemma}
	\newtheorem{corollario}{Corollario}[section]


\theoremstyle{definition}				
	\newtheorem{definizione}{Definizione}[section]%[chapter]
	\newtheorem*{definizione*}{Definizione}	%definizione non numerata
	\newtheorem*{notazione}{Notazione}

\theoremstyle{remark}
	\newtheorem{oss}{Observation}[section]
	\newtheorem*{oss*}{Observation}
	

	% COMANDI CUSTOM:

% Let's try this... 
\NewDocumentCommand{\R}{g g}{
	\IfNoValueTF{#1}
		{\mathbb{R}} % print \mathbb{R} when no arguments are provided
		{
		\mathbb{R}^{
			\IfNoValueTF{#2}
			{#1}	% print \mathbb{R}^{#1} if only one argument is provided
			{#1 \times #2} % else prints \mathbb{R}^{#1\times #2}
		}
		}
	}	

\newcommand{\Diag}[1]{\operatorname{Diag}\left(#1\right)}
\newcommand{\iu}{\dot{\imath}}

	
% ------------------------- INIZIO CODICE -------------------------
\usetheme{Madrid}


\title[ISPR Assignment 4]{AntisymmetricRNN: a dynamical systems view on RNNs}	%ovviamente è provvisorio
\subtitle{Intelligent Systems for Pattern Recognition\\ 4th Assignment} 
\author[Andrea Marino]{Andrea Marino {\smaller (matr. 561935)}}
\institute[DI UniPi]{Università di Pisa}
%\titlegraphic{\includegraphics[width=2cm]{Immagini/cherubino_black.eps}}
\date{\today}

%\AtBeginSection[] 			% Vedi se è opportuno, in una tesi...			
%{
%	\begin{frame}
%		\frametitle{Sommario}
%		\tableofcontents[currentsection,subsectionstyle=show/show/hide] 
%	\end{frame}
%}

\begin{document}
	\begin{frame}[plain]
		\titlepage
	\end{frame}
	
%\section*{Sommario}
%	\setcounter{tocdepth}{1}
%	\begin{frame}
%		\frametitle{Sommario}
%		\tableofcontents
%	\end{frame}
%	
%	\setcounter{tocdepth}{2}
	\begin{frame}{Introduction}
		\begin{block}{}
			The following assignment is based on:

			B. Chang, M. Chen, E. Haber, E. H. Chi, 
			\emph{AntisymmetricRNN: A Dynamical System View on Recurrent Neural Networks}, ICLR 2019

			%All credit goes to the original authors, while any error is to be attributed to my adaptation 
		\end{block}

		\onslide<2->{In this paper, connections between RNNs and ODEs are drawn. 
		Specifically, the issue of RNN trainability is studied within this theoretical framework. 
		The AntisymmetricRNN architecture is proposed to address such issue. %in such context. 
		%Its key design aspects stem from the theory of dynamical systems and that of 
		%leveraging the stability of its underlying ODE to ensure smooth gradient flow.
		}
		\smallskip 

		\onslide<3->{AntisymmetricRNN:}
		\begin{itemize}
			\item<3-> Is able to caputre long-term dependencies, thanks to the stability property of its 
			underlying ODE
			\item<4-> Is an alternative to gated variants of RNNs {\smaller (e.g. LSTMs, GRUs)}
			\item<5-> Is an alternative to constraining the weight matrices to be unitary. 
			%{\smaller (A computationally expensive approach that doesn't fully solve the issue
			%of trainability, because of nonlinear activations.)}
		\end{itemize}
		\medskip

		\onslide<6->{The AntisymmetricRNN can be seen as the forward Euler method 
		applied to an ODE that describes the dynamics of a RNN {\smaller (in the continuum)}.}
		\smallskip

		\onslide<7->{The core concept is that of \alert{stability}. 
		Both the stability of the ODE solution and of the 
		numerical method lead to the key design features that characterize the AntisymmetricRNN.}
		
		
		%What we'll see:
		%\begin{itemize}
		%	\item Stability of \emph{solutions}:
		%	\begin{itemize}
		%		\item Related to backward propagation
		%		\item Unstable solutions necessarily give exploding/vanishing gradients
		%	\end{itemize}
		%	\item Staility of \emph{numerical methods}:
		%	\begin{itemize}
		%		\item stable solutions may be integrated badly by some methods {\smaller 
		%		(a known problem in numerical methods for ODE)}
		%		\item Related to forward propagation
		%		\item A stable method won't amplify the numerical errors that will inevitably
		%		take place
		%	\end{itemize}
		%\end{itemize}
	\end{frame}



	\begin{frame}{Model description}{The (gated) AntisymmetricRNN's equation}
		\begin{block}{AntisymmetricRNN}
			The full equation describing a (gated) AntisymmetricRNN is:
			\[
				\begin{aligned}
					\textcolor<2>{red}{z_t} &= \sigma\qty((W_h-W_h^t-\gamma I)h_{t-1}+V_z x_t+b_z)\\
					h_t &= h_{t-1}\textcolor<3>{red}{+}\epsilon\,\textcolor<2>{red}{z_t\odot}\tanh\qty((W_h-W_h^t-\gamma I)h_{t-1}+V_h x_t+b_h)
				\end{aligned}
			\]
			where $\odot$ denotes the Hadamard product, $\sigma$ denotes the sigmoid function,
			$V_{-}\in\R{n}{p},W_h\in\R{n}{n},b_{-}\in\R{n}$ are model parameters,
			$x_t\in\R{p},h_{t-1}\in\R{n}$ are the input at time $t$ and the previous hidden state (respectively),
			$\epsilon>0,\gamma>0$ are hyperparameters.
		\end{block}
		\begin{itemize}
			\item<2-> \alert<2>{$z_t$}: input gate that controls the flow of information into the hidden states,
				via Hadamard product.
			\item<3-> \alert<3>{$+$}: residual connection
			%\item<4-> \alert<4>{$W_h-W_h^t$}: weight matrix of the 
		\end{itemize}
		
	\end{frame}



	\begin{frame}{Equation 1/3}{}
		The "heart" of the model is given by the equation
		\begin{equation}\label{eqn:RNN_euler}
			h_t = h_{t-1}+\epsilon\,\tanh\qty((W_h-W_h^t-\gamma I)h_{t-1}+V_h x_t+b_h).
		\end{equation}

		\onslide<2->{
			Equation~\eqref{eqn:RNN_euler} comes from the explicit Euler (EE) method applied to the ODE
			\begin{equation}\label{eqn:RNN_ODE}
				h'(t)=\tanh\qty((W_h-W_h^t-\gamma I)h(t)+V_h x(t)+b_h),
			\end{equation}
			\vspace{-\baselineskip}
		}

		\onslide<3->{
			which in turn can be seen as the usual update equation for the hidden state in an RNN,
			but in the continuum.
		}
		\medskip

		\onslide<4->{There are two notable differences, whose role will be explained:}
		\begin{enumerate}
			\item<5-> Skew-symmetric weight matrix $W_h-W_h^t$
			\item<6-> Diffusion term $-\gamma I$
		\end{enumerate}
		\smallskip

		\onslide<7->{\emph{Notation}: $[n]\coloneqq\qty{1,\dots,n}$, 
		$\lambda_k(M)$: $k$-th eigenvalue of $M\in\R{n}{n}$.}
	\end{frame}



	\begin{frame}{Equation 2/3}{Skew-symmetric weight matrix}
		The skew-symmetric matrix $W_h-W_h^t$ comes from the requirement of \emph{stability}
		of the solutions of~\eqref{eqn:RNN_ODE} (leaving aside $-\gamma I$ for now).
		\medskip

		\onslide<2->{In fact, let $A(t)\coloneqq\pdv*{h(t)}{h(0)}$.}
		\onslide*<3>{\begin{oss}
			$A(t)$ represents the jacobian of the hidden state $h(t)$ w.r.t. the initial 
			hidden state $h_0$. Precisely the quantity of interest in BPTT.
		\end{oss}}
		\onslide*<4->{By differentiating~\eqref{eqn:RNN_ODE} w.r.t. $h(0)$ we get}
		\onslide<5->{
			\begin{equation}\label{eqn:RNN_ODE_diff_h}
				A'(t)=J(t)A(t)\qquad\text{with initial condition }A(0)=I,
			\end{equation}
			\vspace*{-\baselineskip}
		}

		\onslide*<6->{where $J(t)$ is the Jacobian of the RHS in~\eqref{eqn:RNN_ODE} w.r.t $h(t)$, given by:
			\[
				J(t)=\Diag{\tanh'\qty((W_h-W_h^t)h(t)+V_hx(t)+b_h)}\cdot\qty(W_h-W_h^t)\equiv D(t)\cdot M.
			\]
		}

		\onslide*<7->{$J(t)$ is a smooth, slowly changing function of $t$, and 
		$\lambda_k\qty(J(t))\in \iu\R\quad\forall\,k\in [n]$.}
		\smallskip 

		\onslide*<8->{Therefore, the solution to~\eqref{eqn:RNN_ODE_diff_h} is 
		$A(t)=\exp(\int_0^t J(s)\dd{s})\cdot A(0)\approx\exp(J\cdot t)$.}
		\onslide*<9->{Consequently, $\lambda_k(A(t))\approx 1\quad\forall k\in[n]$, 
		since $\lambda_k(A(t))=\exp\qty(\lambda_k\qty(J(t)))$%\quad\forall\,k\in[n]$
		.}

		\begin{itemize}
			\item<10-> ODE solutions are \alert{stable}
			\item<11-> Gradient norm is preserved (no gradient explosion/vanish)
		\end{itemize} 
	\end{frame}



	\begin{frame}{Equation 3/3}{Diffusion term}
		%The diffusion term $-\gamma I$ is added to the antisymmetric weight matrix to make 
		%the explicit Euler method stable.

		%The \emph{stability region} of the explicit Euler method is 
		%$\qty{z\in\mathbb{C}\mid \abs{1+z}\le 1}$. 
		%Therefore, 
		The EE method is 
		\alert{stable} if {\smaller $(J_t\equiv J\text{ evaluated in }h_t)$}:
		\[
			\abs{1+\epsilon\cdot\lambda_k\qty(J_t)}\le 1\qquad\forall\,k\in[n]\quad\forall\,t.
		\]
		%\vspace{-1.5em}

		\onslide*<2>{
			\vspace{-\baselineskip}
			\begin{oss}
				Stability of the EE method: numerical errors are not amplified by the 
				forward propagation step {\smaller (a stable solution of an ODE may still be integrated 
				badly by a numerical method)}.
			\end{oss}}
		\onslide*<3->{
			Now,
			%\begin{equation*}
				$\lambda_k\qty(J(t))\in \iu\R\quad\forall\,k\implies\abs{1+\epsilon\cdot\lambda_k\qty(J_t)}>1\quad\forall\,\epsilon>0,\ \forall\,k\in[n].$
			%\end{equation*}
		}

		\onslide*<4->{
			\smallskip
			That is, the condition that allows for long-term dependencies to be preserved conflicts with 
			the stability condition of the forward method.
		}
		\medskip 

		\onslide*<5->{Hence a \emph{diffusion term} is added to the system, 
		to stabilize the numerical method:
		$W_h-W_h^t\leadsto W_h-W_h^t-\gamma I$, and the Jacobian becomes $D(t)\cdot\qty(M-\gamma I)$.}
		\smallskip 

		\begin{itemize}
			\item<6-> With this modification, $\Re\lambda_k\qty(J(t))<0$ slightly 
			%{\smaller (as $\operatorname{Spec}\qty(J(t))$ is shifed by $-\gamma$)}.
			\item<7-> Furthermore, $W_h-W_h^t-\gamma I$ is not skew-symmetric anymore!
		\end{itemize}

		%\begin{oss}
		%	With this modification, $\Re\lambda_k\qty(J(t))<0$ slightly 
		%	(as $\operatorname{Spec}\qty(J(t))$ is shifed by $-\gamma$).
%
		%	Furthermore, $W_h-W_h^t-\gamma I$ is not antisymmetric anymore!
		%\end{oss}
	\end{frame}



	\begin{frame}{Key empirical results}
		%Predictable dynamics: simple example with 2x2 matrices show that the dynamics
		%described by the AntisymmetricRNN is fully predictable, knowing the 
		%eigenvalues of the (randomly initialized) weight matrix. 
		%The dynamics of the vanilla RNN is unpredictable

		\textbf{Permuted pixel-by-pixel MNIST}.

		Input data: sequences of length $28\times 28=784$.
		A (fixed, random) permutation is applied to each sequence.

		\onslide<2->{To succeed, a model must be able to learn non-local,
		 long-range dependencies.} 

		\onslide<3->{AntisymmetricRNN achieves $95.8\%$ accuracy 
		with $128$ units and $10k$ parameters, surpassing LSTM (acc. $92.1\%$, 
		$128$ units, $68k$ parameters) and other models.
		\medskip}


		\onslide<4->{\textbf{Noise-padded CIFAR10}.

		Input data: sequences $\qty{v_i}_{i\in[1000]}$ with $v_i\in\R{32}$. Only the 
		first $32$ steps of the sequence carry information.}

		\onslide<5->{Gated AntisymmetricRNN achieves $54.7\%$ accuracy 
		with $256$ units and $37k$ parameters, 
		surpassing LSTM (acc. $11.6\%$, $128$ units, $69k$ parameters). 
		\smallskip}

		\onslide<6->{A direct study of the mean and stdev of the eigenvalues
		of the end-to-end Jacobian 
		of the networks for different sequence lengths
		show that AntisymmetricRNN keeps them close to 1 (with low stdev), 
		provided that $\gamma$ is modest, even across many time steps.} 
		\onslide<7->{On the other hand, LSTM's eigenvalue decay to $0$.} 
		\begin{itemize}
			\item<8-> LSTM performs badly on the task
			\item<9-> mean $\approx 1$ and stdev $\approx 0$ indicate 
			non-exploding/non-vanishing gradients
		\end{itemize}
	\end{frame}



	\begin{frame}{Conclusions}
		AntisymmetricRNN is a simple yet effective model, which draws inspiration in its design
		from the theory of dynamical systems and that of ODE numerical methods. 
		\medskip 

		\onslide<2->{Main \textbf{pros}:}
		\begin{itemize}
			\item<2-> Few parameters: a skew-symmetric matrix $M\in\R{n}{n}$ only requires $n(n-1)/2$ 
			parameters to be fully specified
			\item<3-> Performance: as the experiments show, the architecture is able to capture long-term 
			dependencies and achieve good results on the tasks
			\item<4-> Little computation overhead, w.r.t. other approaches 
			{\smaller (e.g. unitary matrices)}.
		\end{itemize}

		\onslide<5->{Main \textbf{cons}:}
		\begin{itemize}
			\item<5-> Careful search of hyperparameters: {\smaller the stability of the forward 
			propagation depends on $\epsilon\cdot\lambda\qty(J(t))$, 
			$\lambda\qty(J(t))$ in turn depends on $\gamma$.}
			\item<6-> $\gamma$ hyperparameter imposes a tradeoff:
			\begin{itemize}
				\item<7-> {$\bm{\gamma\approx 0}$}: The weight matrix is close to being skew-symmetric
				$\implies$ stable backward propagation, but less stable forward propagation
				\item<8-> {$\bm{\gamma>0}$}: The Jacobian's eigenvalue have negative real part 
				$\implies$ lossy system (aka long-term dependencies aren't well captured), 
				but the Euler method is more stable (aka forward propagation is more stable). 
			\end{itemize} 
		\end{itemize}
	\end{frame}



\end{document}

