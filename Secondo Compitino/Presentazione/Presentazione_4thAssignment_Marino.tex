\documentclass[10pt,xcolor={table,dvipsnames}]{beamer} 		% carica automaticamente amsthm, amssymb, amsmath, graphicx

\usepackage[T1]{fontenc}
%\usepackage[utf8]{inputenc}
\usepackage[italian]{babel}

\usepackage{mathtools}					% amsmath sotto steroidi
\usepackage{mathdots}

\usepackage{mathrsfs}					% Per dei caratteri matematici migliori: \mathscr{} e \mathcal{}
%\usepackage{braket} 					% Per il comando \Set, e altre (poche) cose
\usepackage[italian]{varioref}			% Per usare il comando \vref{label}, che dà dei collegamenti più dettagliati
\usepackage{microtype}					% Migliora la tipografia, permettendo ad alcuni elementi di sporgere leggermente
%\usepackage{textcomp}					% Dovrebbe aggiungere più simboli

\usepackage{relsize}					% Per usare \mathbigger{} ecc

\usepackage{multirow}					% per usare il comando multirow
\usepackage{tabularx}					% per fare tabelle. Carica il pacchetto array, per gli array.
%\usepackage{arydshln}					% per le linee tratteggiate nelle tabelle

%\usepackage[many]{tcolorbox}

\usepackage{hyperref}					% hyperref improved??




\hypersetup{colorlinks, linkcolor=blue}

\DeclarePairedDelimiter{\absval}{\lvert}{\rvert}
\DeclarePairedDelimiter{\norma}{\lVert}{\rVert}

%\setcounter{tocdepth}{1}	% profondità dell'indice

	% TEOREMI CUSTOM:
\theoremstyle{plain}					% Definisce ambienti per Teoremi, esercizi, corollari... Con lo stile adeguato
	\newtheorem{proposizione}{Proposizione}[section]
	\newtheorem*{proposizione*}{Proposizione}
	
	\newtheorem{teorema}{Teorema}[section]
	\newtheorem*{teorema*}{Teorema}
		
	%\newtheorem{lemma_es}{Lemma}[esercizio]
	%\newtheorem{lemma}{Lemma}[section]
	\newtheorem*{lemma*}{Lemma}
	\newtheorem{corollario}{Corollario}[section]


\theoremstyle{definition}				
	\newtheorem{definizione}{Definizione}[section]%[chapter]
	\newtheorem*{definizione*}{Definizione}	%definizione non numerata
	\newtheorem*{notazione}{Notazione}

\theoremstyle{remark}
	\newtheorem{oss}{Osservazione}[section]
	\newtheorem*{oss*}{Osservazione}
	
	% COLORI CUSTOM
\definecolor{pythonlb}{RGB}{156, 220, 254}

	% COMANDI CUSTOM
\newcommand{\teal}[1]{\textcolor{teal}{#1}}
\newcommand{\plb}[1]{\textcolor{pythonlb}{#1}}

	
% ------------------------- INIZIO CODICE -------------------------
\usetheme{Madrid}
%\setbeamercolor{blockcolor}{bg=UniWhite, fg=DMDarkBlue}

\title[ISPR - 4th Assignment]{Assignment 4}
\subtitle{Bayesian Network implementation} 
\author[Andrea Marino]{Andrea Marino \small{(matr. 561935)}}
\institute[DI UniPi]{Università di Pisa, dipartimento di Informatica}
%\titlegraphic{\includegraphics[width=2cm]{Immagini/cherubino_black.eps}}
\date{\today}


\begin{document}
	\begin{frame}[plain]
		\titlepage
	\end{frame}

	\begin{frame}{Code structure}
		%The code for this assignment (well commented) is available
		%\href{https://github.com/M4rinz/ISPR-Assignments}{on GitHub}.
		%In this presentation, only the most essential aspects will be described.

		\onslide<2->{The bulk of the project is contained in the \texttt{bayes\_net.py}
		and \texttt{distributions.py} files. }
		\onslide<3->{There are other accessory files, as well as small tests. These are:}
		\begin{itemize}
			\item<3-> \texttt{BNTypes.py}: contains the declarations of the principal 
			datatypes\footnote{Even if not required by Python,
			they're helpful for e.g. debugging purposes and code clarity} used for the project. These are:
			\begin{itemize}
				\item<4-> \texttt{\plb{Arc} =
					\teal{Tuple}[\teal{str}|\teal{int},
					\teal{str}|\teal{int}]
					}, for the arcs of the network. 
					The first element represents the tail node, 
					the second one the head node. 
					Either the node's label or its ID can be specified.  
				\item<5-> \texttt{\plb{P} = \teal{float}|\plb{List}[\teal{float}]}, 
				the parameters of the distributions (Bernoulli of categorical). 
				Can be seen as the "column" of the CPT.
				\item<6-> \texttt{\plb{PassedConditions} = \teal{frozenset}[\plb{List}[\teal{Tuple}[\teal{str},\teal{int}]]]}, 
				the type of the rows of the CPT. 
			\end{itemize}
			\onslide<7->{The last two will be further commented in later sections.}
			\item<8-> \texttt{exceptions.py}: contains some custom exceptions that are 
			used for some error management (mainly for debugging purposes)
			\item<9-> \texttt{Tests} folder: small programs to test the functionality
			of the classes.
		\end{itemize}
	\end{frame}

	\begin{frame}{The \texttt{BayesNetwork} class}{}
		
	\end{frame}


    
    \begin{frame}{Conclusion}
        We've outlined the structure of the code produced for this assignment, 
		focusing on the most relevant aspects. 
		\bigskip 


		The code for this assignment (well commented) is available
		\href{https://github.com/M4rinz/ISPR-Assignments}{on GitHub}.
    \end{frame}

\end{document}

