\documentclass[10pt,xcolor={table,dvipsnames}]{beamer} 		% carica automaticamente amsthm, amssymb, amsmath, graphicx

\usepackage[T1]{fontenc}
%\usepackage[utf8]{inputenc}
\usepackage[italian]{babel}

\usepackage{mathtools}					% amsmath sotto steroidi
\usepackage{mathdots}
\usepackage{physics}

\usepackage{mathrsfs}					% Per dei caratteri matematici migliori: \mathscr{} e \mathcal{}
%\usepackage{braket} 					% Per il comando \Set, e altre (poche) cose
\usepackage[italian]{varioref}			% Per usare il comando \vref{label}, che dà dei collegamenti più dettagliati
\usepackage{microtype}					% Migliora la tipografia, permettendo ad alcuni elementi di sporgere leggermente
%\usepackage{textcomp}					% Dovrebbe aggiungere più simboli

\usepackage{relsize}					% Per usare \mathbigger{} ecc

\usepackage{multirow}					% per usare il comando multirow
\usepackage{tabularx}					% per fare tabelle. Carica il pacchetto array, per gli array.
%\usepackage{arydshln}					% per le linee tratteggiate nelle tabelle
\usepackage{tikz}

%\usepackage[many]{tcolorbox}

\usepackage{hyperref}					% hyperref improved??




\hypersetup{colorlinks, linkcolor=blue}

\DeclarePairedDelimiter{\absval}{\lvert}{\rvert}
\DeclarePairedDelimiter{\norma}{\lVert}{\rVert}

%\setcounter{tocdepth}{1}	% profondità dell'indice

	% TEOREMI CUSTOM:
\theoremstyle{plain}					% Definisce ambienti per Teoremi, esercizi, corollari... Con lo stile adeguato
	\newtheorem{proposizione}{Proposizione}[section]
	\newtheorem*{proposizione*}{Proposizione}
	
	\newtheorem{teorema}{Teorema}[section]
	\newtheorem*{teorema*}{Teorema}
		
	%\newtheorem{lemma_es}{Lemma}[esercizio]
	%\newtheorem{lemma}{Lemma}[section]
	\newtheorem*{lemma*}{Lemma}
	\newtheorem{corollario}{Corollario}[section]


\theoremstyle{definition}				
	\newtheorem{definizione}{Definizione}[section]%[chapter]
	\newtheorem*{definizione*}{Definizione}	%definizione non numerata
	\newtheorem*{notazione}{Notazione}

\theoremstyle{remark}
	\newtheorem{oss}{Osservazione}[section]
	\newtheorem*{oss*}{Osservazione}
	
	% COLORI CUSTOM
\definecolor{pythonlb}{RGB}{156, 220, 254}
\definecolor{pythondb}{RGB}{48, 116, 208}

	% COMANDI CUSTOM
\newcommand{\ttteal}[1]{\texttt{\textcolor{teal}{#1}}}
\newcommand{\teal}[1]{\textcolor{teal}{#1}}
\newcommand{\plb}[1]{\textcolor{pythonlb}{#1}}
\newcommand{\pdb}[1]{\textcolor{pythondb}{#1}}


	
% ------------------------- INIZIO CODICE -------------------------
\usetheme{Madrid}
%\setbeamercolor{blockcolor}{bg=UniWhite, fg=DMDarkBlue}

\title[ISPR - 4th Assignment]{Assignment 4}
\subtitle{Bayesian Network implementation} 
\author[Andrea Marino]{Andrea Marino \small{(matr. 561935)}}
\institute[DI UniPi]{Università di Pisa, dipartimento di Informatica}
%\titlegraphic{\includegraphics[width=2cm]{Immagini/cherubino_black.eps}}
\date{\today}


\begin{document}
	\begin{frame}[plain]
		\titlepage
	\end{frame}


\section*{Code structure}
	\begin{frame}{Code structure}{Description of the project's files}	%RIVEDI!
		The code for this assignment (well commented) is available
		\href{https://github.com/M4rinz/ISPR-Assignments}{on GitHub}.
		In this presentation the focus is on the most relevant aspects.
		\medskip

		\onslide<2->{The bulk of the project is contained in the \texttt{bayes\_net.py}
		and \texttt{distributions.py} files.}
		\onslide<3->{The former contains the classes used
		to define the \emph{structure} of the network, the latter the classes 
		used to assign \emph{distributions} to the nodes.}
		\smallskip

		\onslide<4->{There are other accessory files, as well as small tests. These are:}
		\begin{itemize}
			\item<4-> \texttt{BNTypes.py}: contains the declarations of the principal 
			datatypes used for the project 
			(helpful for, e.g., code clarity even if not required by Python)
			%These are:
			%\begin{itemize}
			%	\item<3-> \texttt{\plb{Arc} =
			%		\teal{Tuple}[\teal{str}|\teal{int},
			%		\teal{str}|\teal{int}]
			%		}, for the arcs of the network. 
			%		The first element represents the tail node, 
			%		the second one the head node. 
			%		Either the node's label or its ID can be specified.  
			%	\item<4-> \texttt{\plb{P} = \teal{float}|\plb{List}[\teal{float}]}, 
			%	the parameters of the distributions (Bernoulli of categorical). 
			%	Can be seen as the "column" of the CPT.
			%	\item<5-> \texttt{\plb{PassedConditions} = \teal{frozenset}[\plb{List}[\teal{Tuple}[\teal{str},\teal{int}]]]}, 
			%	the type of the rows of the CPT. 
			%\end{itemize}
			%\onslide<6->{The last two will be further commented in later sections.}
			\item<5-> \texttt{exceptions.py}: contains some custom exceptions that are 
			used for some error management (mainly for debugging purposes)
			\item<6-> \texttt{Tests} folder: small programs to test the functionality
			of the classes.
		\end{itemize}
		\medskip

		\onslide<7->{A Bayesian Network is an object of the \ttteal{BayesNetwork} class.
		Inspiration has been taken from the formal definition of a graph 
		as a pair $(V,E)$ of vertices and edges: the main attributes of the class
		are a list of nodes and a list of arcs.}
	\end{frame}


%\section*{BayesNetwork class}
	%\begin{frame}{The \texttt{BayesNetwork} class 1/2}{Main attributes and methods}
	%	To implement a Bayesian Network, I took inspiration from the formal 
	%	definition of a graph $G$ as a pair $(V,E)$ of vertices and edges.
	%	\smallskip 
	%
	%	\onslide<2->{\ttteal{Bayesnetwork}'s main attributes:}
	%	\begin{itemize}
	%		\item<2-> \texttt{\pdb{self}.\_nodes\_list}: a list of the nodes in the network.
	%		Nodes are objects of the class \ttteal{Node} (described later).
	%		\item<3-> \texttt{\pdb{self}.\_arcs\_list}: a list of the arcs in the network.
	%	\end{itemize}
	%
	%	\onslide<4->{\ttteal{Bayesnetwork}'s main methods:}
	%	\begin{itemize}
	%		\item<4-> \texttt{\pdb{self}.add\_nodes}: adds any number of nodes 
	%		provided as argument to the network.
	%		\item<5-> \texttt{\pdb{self}.add\_arcs}: adds any number of arcs provided
	%		as argument to the network. 
	%
	%		The tail node is added as a parent of the head node, and the head 
	%		node is added as a child of the head node. This is done by exploting
	%		methods of the \ttteal{Node} class. 
	%		\smallskip
	%
	%		The function also checks that arcs are added correctly by the user 
	%		(e.g. no self loops).
	%	\end{itemize}
	%\end{frame}

	%\begin{frame}{The \texttt{BayesNetwork} class 2/2}{Main characteristics}
	%	If a list of nodes and a list of arcs is passed to the class constructor,
	%	a Bayesian Network will be initialized with such nodes and arcs. 
	%	Otherwise nodes and arcs can be added at any moment with the aforementioned 
	%	methods.
	%	\smallskip
	%
	%	\onslide<2->{Only addition of nodes and arcs is supported. 
	%	Methods to safely remove nodes and arcs haven't been provided, as these 
	%	delicate operations weren't seen as necessary.}
	%	\smallskip
	%
	%	\textbf{\textcolor{red}{TODO!!}} When inserting an arc, it is checked 
	%	that the graph is acyclic.
	%\end{frame}


\section*{Node class}
	\begin{frame}{The \texttt{Node} class}{Main attributes and methods}
		Each node in the network is an object of the class \ttteal{Node}.
		
		\onslide<2->{Main attributes:}
		\begin{itemize}
			\item<2-> \texttt{\pdb{self}.label}: to describe (and identify) each node 
			\item<3-> \texttt{\pdb{self}.FS} and \texttt{\pdb{self}.BS}: the list of
			(respectively) children and parents of the node
			\item<4-> \texttt{\pdb{self}.distribution}: the distribution of the 
			random variable associated to the node. 
			It's an instance of the class \ttteal{Prior}
			or \ttteal{CPT}. 
		\end{itemize}

		\onslide<5->{Main methods:}
		\begin{itemize}
			\item<5-> \texttt{\pdb{self}.assign\_CPT}: assigns a distribution to 
			the node. Based on the datatype of the provided initialization, 
			it will automatically assign a conditional or unconditional distribution 
			(so it's not necessary to create an object of the class \ttteal{Prior} or \ttteal{CPT})
			\item<6-> \texttt{\pdb{self}.\_add\_to\_star}: adds a node to the 
			forward/backward star. When an arc is added to the graph, the 
			tail/head nodes are automatically added as parents/children through this method
			\item<7-> \texttt{\pdb{self}.print\_attributes}: prints a summary of the most 
			relevant information of the node.
		\end{itemize}
	\end{frame}

	%\begin{frame}{The \texttt{Node} class 2/2}{Other slide?}
	%	Altri discorsi, se serve (credo che serva)
	%\end{frame}


\section*{Probabilities}
	\begin{frame}{Distributions of the nodes' random variables}
		{Broad description, \texttt{Prior} class' sampling method}
		It's been decided to manage the distributions separately from the 
		main structure of the network. 
		\smallskip 

		\onslide<2->{The classes \ttteal{Prior} and \ttteal{CPT} are used to 
		represent unconditioned and conditioned distributions respectively, and
		can manage both Bernoullian and categorical random variables.}

		\onslide<3->{The \ttteal{Prior}'s constructor accepts a parameter \texttt{\plb{p}}:
		if $\mathtt{\plb{p}}\in[0,1]$ then the r.v. is Bernoullian, if \texttt{\plb{p}}
		is a stochastic vector then the r.v. is categorical (similarly for \ttteal{CPT}).}
		\smallskip

		\onslide<4->{The two classes have similar methods, the main one being 
		\texttt{\pdb{self}.sample}. The basic mechanism for sampling is the same
		behind both classes.}

		\onslide<5->{For the \ttteal{Prior} class: $\mathtt{\plb{p}}=\qty(p_1,\dots,p_N)$.}
		\begin{enumerate}
			\item<5-> A random number $r\in[0,1]$ is generated
			\item<6-> The sampled class is the index $\hat{\imath}\in\qty{1,\dots,N}$
			s.t. $p_{\hat{\imath}-1}\le r\le p_{\hat{\imath}}$ $(p_0\coloneqq0)$
			\item<7-> In case $N=1$ (Bernoullian r.v.), the same procedure 
			is used, with slight modifications to the points above.
		\end{enumerate}

		\onslide<8->{Other class methods include: methods for returning, 
		changing the \texttt{\plb{p}} parameter, methods to print the 
		distribution\dots}
	\end{frame}

	\begin{frame}{The \texttt{CPT} class}
		The \ttteal{CPT} class is more complicated than the \ttteal{Prior} one.
		\smallskip 

		\onslide<2->{The main attribute is \texttt{\pdb{self}.cond\_distrib},
		which represents the conditional distribution. It's a dictionary with:}
		\begin{itemize}
			\item<2-> \textbf{keys} which represent the rows of the CPT. 
			The keys cover the whole possible instantiations of the conditioning side.

			\onslide<3->{The datatype for a key is 
			\texttt{\teal{frozenset}[\teal{Tuple}[\teal{Node},\teal{int}]]}.
			\ttteal{frozenset}s allow the parents' ordering to be irrelevant.
			This is very helpful for declaration and sampling}
			\item<4-> \textbf{values}, the parameter(s) \texttt{\plb{p}} of the
			distribution (like what's in the \ttteal{Prior} class). 
		\end{itemize}

		\onslide<5->{Main methods:}
		\begin{itemize}
			\item<5-> \texttt{\pdb{self}.build\_cond\_distrib}: replaces the 
			label of each parent from the dictionary passed as argument with
			the \ttteal{Node} object (for easier sampling)
			\item<6-> \texttt{\pdb{self}.sample}. Samples (recursively) a 
			value from each of the parents (up to calling the \texttt{sample}
			method of the \ttteal{Prior} class), 
			\onslide<7->{the sampled values are used to "fill up" 
			the row of the CPT (i.e. fully specify the conditioning side).}
			\onslide<8->{The parameter \texttt{\plb{p}} of the distribution is 
			retrieved, and sampling can be performed according to the value(s) of 
			\texttt{\plb{p}} as described earlier.}
		\end{itemize}
	\end{frame}


\section*{The concert Bayesian Network}
	\begin{frame}{The concert Bayesian Network}{A brief explaination}
		Un po' il raccontino
	\end{frame}

	\begin{frame}{Visual representation of the Network}
		\begin{center}
			\begin{figure}
				\begin{tikzpicture}[scale=0.14]
					\tikzstyle{every node}+=[inner sep=0pt]
					\draw [black] (9.5,-13) circle (3);
					\draw (9.5,-13) node {$S$};
					\draw [black] (21.6,-13) circle (3);
					\draw (21.6,-13) node {$Sc$};
					\draw [black] (21.6,-13) circle (2.4);
					\draw [black] (60.7,-13) circle (3);
					\draw (60.7,-13) node {$W$};
					\draw [black] (60.7,-13) circle (2.4);
					\draw [black] (34.3,-13) circle (3);
					\draw (34.3,-13) node {$Ce$};
					\draw [black] (18.9,-28.2) circle (3);
					\draw (18.9,-28.2) node {$B$};
					\draw [black] (27,-42.3) circle (3);
					\draw (27,-42.3) node {$CH$};
					\draw [black] (34.3,-25.4) circle (3);
					\draw (34.3,-25.4) node {$D$};
					\draw [black] (42.2,-41.8) circle (3);
					\draw (42.2,-41.8) node {$G$};
					\draw [black] (40.1,-55.5) circle (3);
					\draw (40.1,-55.5) node {$Cs$};
					\draw [black] (60.7,-28.2) circle (3);
					\draw (60.7,-28.2) node {$Gpc$};
					\draw [black] (60.7,-41.2) circle (3);
					\draw (60.7,-41.2) node {$M$};
					\draw [black] (57.7,-13) -- (37.3,-13);
					\fill [black] (37.3,-13) -- (38.1,-13.5) -- (38.1,-12.5);
					\draw [black] (11.08,-15.55) -- (17.32,-25.65);
					\fill [black] (17.32,-25.65) -- (17.33,-24.71) -- (16.48,-25.23);
					\draw [black] (21.08,-15.95) -- (19.42,-25.25);
					\fill [black] (19.42,-25.25) -- (20.06,-24.55) -- (19.07,-24.37);
					\draw [black] (20.39,-30.8) -- (25.51,-39.7);
					\fill [black] (25.51,-39.7) -- (25.54,-38.76) -- (24.67,-39.25);
					\draw [black] (33.11,-28.15) -- (28.19,-39.55);
					\fill [black] (28.19,-39.55) -- (28.97,-39.01) -- (28.05,-38.61);
					\draw [black] (29.11,-44.43) -- (37.99,-53.37);
					\fill [black] (37.99,-53.37) -- (37.78,-52.45) -- (37.07,-53.16);
					\draw [black] (41.75,-44.77) -- (40.55,-52.53);
					\fill [black] (40.55,-52.53) -- (41.17,-51.82) -- (40.18,-51.67);
					\draw [black] (35.6,-28.1) -- (40.9,-39.1);
					\fill [black] (40.9,-39.1) -- (41,-38.16) -- (40.1,-38.59);
					\draw [black] (60.7,-31.2) -- (60.7,-38.2);
					\fill [black] (60.7,-38.2) -- (61.2,-37.4) -- (60.2,-37.4);
					\draw [black] (58.24,-42.91) -- (42.56,-53.79);
					\fill [black] (42.56,-53.79) -- (43.51,-53.74) -- (42.94,-52.92);
					\draw [black] (58.89,-30.59) -- (41.91,-53.11);
					\fill [black] (41.91,-53.11) -- (42.79,-52.77) -- (41.99,-52.17);
					\draw [black] (58.44,-14.97) -- (29.26,-40.33);
					\fill [black] (29.26,-40.33) -- (30.2,-40.18) -- (29.54,-39.43);
					\draw [black] (32.16,-15.11) -- (21.04,-26.09);
					\fill [black] (21.04,-26.09) -- (21.96,-25.89) -- (21.25,-25.17);
				\end{tikzpicture}
			\label{fig:BNConcert}
			\caption{\emph{Graphical representation of the concert's Bayesian
			Network. The nodes with the double circles have a categorical distribution.}}
		\end{figure}
		\end{center}
	\end{frame}

    
\section*{Comments and conclusion}
    \begin{frame}{Conclusion, and comments}
        \textbf{\textcolor{red}{TO FINISH!!}} We've outlined the structure of the code produced for this assignment, 
		focusing on the most relevant aspects. 
		\smallskip 

		\onslide<2->{Programming was instructive and intriguing, although not
		very challenging. Having good data structures and an elaborate 
		mechanism of bells and whistles revealed to be useful.}
		\smallskip 

		\onslide<3->{Conceiving the model was very stimulating. 
		Bayesian Networks look powerful at first sight, giving the 
		impression of being able to model many phenomena from the real word. 
		But by working with them, one realizes that they fall short really 
		soon, and that limitations in terms of expressiveness soon show up.}
		\smallskip 

		\onslide<4->{One example from the concert BN: the civil engineering 
		office's response's dependency from the weather is an irrealistic 
		simplification. More properly, it should depend on a weather forecast. 
		But it's complicated to model the relation between the actual weather 
		and the weather forecast in a Bayesian Network: the two variables 
		are not independent (of course), on the other hand a distribution 
		conditioned on the true weather is not 
		a good model (i.e. representation) for the weather forecast.}
	\end{frame}

\end{document}

