\documentclass[10pt,xcolor={table,dvipsnames}]{beamer} 		% carica automaticamente amsthm, amssymb, amsmath, graphicx

\usepackage[T1]{fontenc}
%\usepackage[utf8]{inputenc}
\usepackage[italian]{babel}

\usepackage{mathtools}					% amsmath sotto steroidi
\usepackage{mathdots}

\usepackage{mathrsfs}					% Per dei caratteri matematici migliori: \mathscr{} e \mathcal{}
%\usepackage{braket} 					% Per il comando \Set, e altre (poche) cose
\usepackage[italian]{varioref}			% Per usare il comando \vref{label}, che dà dei collegamenti più dettagliati
\usepackage{microtype}					% Migliora la tipografia, permettendo ad alcuni elementi di sporgere leggermente
%\usepackage{textcomp}					% Dovrebbe aggiungere più simboli

\usepackage{relsize}					% Per usare \mathbigger{} ecc

\usepackage{multirow}					% per usare il comando multirow
\usepackage{tabularx}					% per fare tabelle. Carica il pacchetto array, per gli array.
%\usepackage{arydshln}					% per le linee tratteggiate nelle tabelle

%\usepackage[many]{tcolorbox}

\usepackage{hyperref}					% hyperref improved??




\hypersetup{colorlinks, linkcolor=blue}

\DeclarePairedDelimiter{\absval}{\lvert}{\rvert}
\DeclarePairedDelimiter{\norma}{\lVert}{\rVert}

%\setcounter{tocdepth}{1}	% profondità dell'indice

	% TEOREMI CUSTOM:
\theoremstyle{plain}					% Definisce ambienti per Teoremi, esercizi, corollari... Con lo stile adeguato
	\newtheorem{proposizione}{Proposizione}[section]
	\newtheorem*{proposizione*}{Proposizione}
	
	\newtheorem{teorema}{Teorema}[section]
	\newtheorem*{teorema*}{Teorema}
		
	%\newtheorem{lemma_es}{Lemma}[esercizio]
	%\newtheorem{lemma}{Lemma}[section]
	\newtheorem*{lemma*}{Lemma}
	\newtheorem{corollario}{Corollario}[section]


\theoremstyle{definition}				
	\newtheorem{definizione}{Definizione}[section]%[chapter]
	\newtheorem*{definizione*}{Definizione}	%definizione non numerata
	\newtheorem*{notazione}{Notazione}

\theoremstyle{remark}
	\newtheorem{oss}{Osservazione}[section]
	\newtheorem*{oss*}{Osservazione}
	
	% COLORI CUSTOM
\definecolor{pythonlb}{RGB}{156, 220, 254}
\definecolor{pythondb}{RGB}{48, 116, 208}

	% COMANDI CUSTOM
\newcommand{\ttteal}[1]{\texttt{\textcolor{teal}{#1}}}
\newcommand{\teal}[1]{\textcolor{teal}{#1}}
\newcommand{\plb}[1]{\textcolor{pythonlb}{#1}}
\newcommand{\pdb}[1]{\textcolor{pythondb}{#1}}


	
% ------------------------- INIZIO CODICE -------------------------
\usetheme{Madrid}
%\setbeamercolor{blockcolor}{bg=UniWhite, fg=DMDarkBlue}

\title[ISPR - 4th Assignment]{Assignment 4}
\subtitle{Bayesian Network implementation} 
\author[Andrea Marino]{Andrea Marino \small{(matr. 561935)}}
\institute[DI UniPi]{Università di Pisa, dipartimento di Informatica}
%\titlegraphic{\includegraphics[width=2cm]{Immagini/cherubino_black.eps}}
\date{\today}


\begin{document}
	\begin{frame}[plain]
		\titlepage
	\end{frame}


\section*{Code structure}
	\begin{frame}{Code structure}{Accessory files}
		The bulk of the project is contained in the \texttt{bayes\_net.py}
		and \texttt{distributions.py} files.

		\onslide<2->{There are other accessory files, as well as small tests. These are:}
		\begin{itemize}
			\item<2-> \texttt{BNTypes.py}: contains the declarations of the principal 
			datatypes\footnote{Even if not required by Python,
			they're helpful for e.g. debugging purposes and code clarity} used for the project. 
			These are:
			\begin{itemize}
				\item<3-> \texttt{\plb{Arc} =
					\teal{Tuple}[\teal{str}|\teal{int},
					\teal{str}|\teal{int}]
					}, for the arcs of the network. 
					The first element represents the tail node, 
					the second one the head node. 
					Either the node's label or its ID can be specified.  
				\item<4-> \texttt{\plb{P} = \teal{float}|\plb{List}[\teal{float}]}, 
				the parameters of the distributions (Bernoulli of categorical). 
				Can be seen as the "column" of the CPT.
				\item<5-> \texttt{\plb{PassedConditions} = \teal{frozenset}[\plb{List}[\teal{Tuple}[\teal{str},\teal{int}]]]}, 
				the type of the rows of the CPT. 
			\end{itemize}
			\onslide<6->{The last two will be further commented in later sections.}
			\item<7-> \texttt{exceptions.py}: contains some custom exceptions that are 
			used for some error management (mainly for debugging purposes)
			\item<8-> \texttt{Tests} folder: small programs to test the functionality
			of the classes.
		\end{itemize}
	\end{frame}


\section*{BayesNetwork class}
	\begin{frame}{The \texttt{BayesNetwork} class 1/2}{Main attributes and methods}
		To implement a Bayesian Network, I took inspiration from the formal 
		definition of a graph $G$ as a pair $(V,E)$ of vertices and edges.
		\smallskip 

		\onslide<2->{\ttteal{Bayesnetwork}'s main attributes:}
		\begin{itemize}
			\item<2-> \texttt{\pdb{self}.\_nodes\_list}: a list of the nodes in the network.
			Nodes are objects of the class \ttteal{Node} (described later).
			\item<3-> \texttt{\pdb{self}.\_arcs\_list}: a list of the arcs in the network.
		\end{itemize}

		\onslide<4->{\ttteal{Bayesnetwork}'s main methods:}
		\begin{itemize}
			\item<4-> \texttt{\pdb{self}.add\_nodes}: adds any number of nodes 
			provided as argument to the network.
			\item<5-> \texttt{\pdb{self}.add\_arcs}: adds any number of arcs provided
			as argument to the network. 

			The tail node is added as a parent of the head node, and the head 
			node is added as a child of the head node. This is done by exploting
			methods of the \ttteal{Node} class. 
			\smallskip

			The function also checks that arcs are added correctly by the user 
			(e.g. no self loops).
		\end{itemize}
	\end{frame}

	\begin{frame}{The \texttt{BayesNetwork} class 2/2}{Main characteristics}
		If a list of nodes and a list of arcs is passed to the class constructor,
		a Bayesian Network will be initialized with such nodes and arcs. 
		Otherwise nodes and arcs can be added at any moment with the aforementioned 
		methods.
		\smallskip

		\onslide<2->{Only addition of nodes and arcs is supported. 
		Methods to safely remove nodes and arcs haven't been provided, as these 
		delicate operations weren't seen as necessary.}
		\smallskip

		\textbf{\textcolor{red}{TODO!!}} When inserting an arc, it is checked 
		that the graph is acyclic.
	\end{frame}


\section*{Node class}
	\begin{frame}{The \texttt{Node} class 1/2}{Main attributes and methods}
		Attributi e metodi principali
	\end{frame}

	\begin{frame}{The \texttt{Node} class 2/2}{Other slide?}
		Altri discorsi, se serve (credo che serva)
	\end{frame}


\section*{Probabilities}
	\begin{frame}{Distributions of the nodes' random variables}
		Una slide un po' spiegona
	\end{frame}

	\begin{frame}{The \texttt{Prior} class}
		Una slide al massimo. Dai.
	\end{frame}

	\begin{frame}{The \texttt{CPT} class 1/2}
		
	\end{frame}

	\begin{frame}{The \texttt{CPT} class 2/2}
		
	\end{frame}

    
    \begin{frame}{Conclusion}
        We've outlined the structure of the code produced for this assignment, 
		focusing on the most relevant aspects. 
		\bigskip 


		The code for this assignment (well commented) is available
		\href{https://github.com/M4rinz/ISPR-Assignments}{on GitHub}.
    \end{frame}

\end{document}

